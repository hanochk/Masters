%!TEX root = thesis.tex
\chapter{Background} \label{ch:background}

In this chapter, we introduce laser scanning and the 3D-reconstruction pipeline. In section~\ref{sec:heritage} we describe our problem's heritage preservation context. In section~\ref{sec:scanners} we introduce 3d scanning followed by the process by which raw scanner data transformed into a model in section~\ref{sec:pipeline}. We then discuss the cleaning process in section~\ref{sec:cleaning}.




\section{Digital heritage preservation} \label{sec:heritage}

Architectural heritage sites in many parts of the world are under threat of deterioration and destruction especially in developing countries. There is a need to preserve these heritage sites due to their historical relevance, if not physically, at least digitally. The demolition of  the Buddhas of Bamiyan by the Taliban in 2001 \cite{Toubekis2009} is a good illustration of this need. Preservation efforts have used various techniques to record historical sites. Early efforts used tape measures and theodolites to produce simple ground plans. More recently photogrammetry allowed geomaticians to produce 3D models \cite{Heritage}. Now laser range scanners allows us to create extremely high resolution 3D model.


\section{3D scanning} \label{sec:scanners}

There is an enormous variety of 3D scanners available on the market today. Each scanner has properties that make them more of less suitable for various scanning tasks. Generally the size of the object and the level of detail one wishes to capture dictates the technology. There are also other considerations, including imaging speed, portability and cost.

Scanning technologies can generally be classified into 2 categories, namely triangulation and time of flight scanners.

Triangulation scanners, as the name suggests, uses trigonometric triangulation to locate of a point in a scene. Triangulation can be performed by either emitting a laser light or by projecting a series of linear patterns and then picking up the reflection with a sensor at a known position. During laser triangulation, the displacement of the object affects the angle at which the beam is returned. When using structured light, the distortions in the patterns from the sensor's perspective can be used to determine the angles of reflected light. Given the distance between the light source and sensor and as well as the angle of emitted and reflection light, Pythagorean theorem can be applied to location of a point on an surface relative to the scanner.

Triangulation scanners are typically used for focal distances 




TOF shoot a laser and times the return
Phase based scanners are also TOF. It 


Triangulation

TOF

Phase


Sometimes an object will fall inbetween the technologies


Small objects require accuracy over a short range. The degree of accuracy depends on the level of detail one wishes to capture. Very dense sampling is required to capture submilimetre objects. Conoscopic holography provides


While this technology may be considered stable on a quite large variety of ”stan- dard" objects, there still exist many complex objects which still require lot of planning, tweaking and specific work, in order to obtain usable and high quality results



Scanners:
	Types
		conoscopy, submillimetric object size
		TOF
		Phase shifting TOF, 1 to 20 meter sized objects
		Laser Triangulation, small to human sized objects


	Speed
	Resolution
	Accuracy
	Range







 Time of flight scanners are generally used in the cultural heritage domain because of their long range (up to kilometers).

Time of flight scanners work by sending out laser pulses that reflect off objects and is then recorded by a sensor at the source. The time it takes to the pulse to return, along with the angle at which the pulse was fired, is used to determine the position of a point on a surface. There is more measurement error (millimeters) along the direction of the laser than the other two axis. This is due to measurement inaccuracies when timing the laser's travel time. Triangulation scanners are more accurate along the direction of the beam (tens of micrometers), but they have a shorter range compared to time of flight scanners. This makes them less popular when people try minimize the number of scans required to capture a site. (Other types of range scanners include structure from motion and structured light.)

Time of flight scanners can be set to sample at different resolutions. Low resolution scans (10 000 points) may take seconds while higher resolution scans (millions of points) may take minutes depending on the model. All scanners return the X,Y and Z coordinates per sample. Most also return an intensity value of the reflected light. More expensive scanners are also mounted with a camera that associates RGB values with every sample. This makes it easier to texture (texture) the final model. Measuring the orientation and GPS position of the scanner is another optional feature that can ease registration process (discussed later).

(Mention Averaged pixel on edges for time of flight)
(Mention multiple sample pulse scanners)
(Mention non uniform point density somewhere? too obvious?)
(Mention scanner resolution increasing over time?)

\section{3D reconstruction pipeline} \label{sec:pipeline}

Some processing is required in order to create a 3d model from a collection of laser range scans. Unwanted objects need to be removed or cleaned. Holes created by removals and occlusions need to be filled. The scans need to registered. Registration is the process of aligning all the scans on a common coordinate system (mention Iterative closest points?).Then the registered point sets have to turned into a surface model. This is also called meshing. Finally the model has to be textured. RGB data from the scanner can be used if it available. Alternatively photos from the site has to be mapped to the model. The second approach is somewhat more time consuming.

The cleaning step can be omitted but the quality of the model will be affected. Objects such as trees or grass do not always mesh nicely, and floating bits of people as they walk around may not be aesthetically pleasing. Hole filling is also not strictly necessary. It can even be argued that it compromises the integrity of the historical record as the filled area would be fabricated.

The first 3 step also do not have to happen in order. It is often the case that hole filling, cleaning and registration tasks are interspersed. Cleaning may be detrimental to the reregistration process as useful correspondences may be removed. Cleaning scans after registration can however be problematic. If the scans have been merged into a single point cloud, loading the scan into main memory may not be an option on some programs. One also loses the 2d grid structure of the scan after merging. The scan's grid allows one to interpret the scan as a 2d image which may be easier to clean.


\section{Point cloud cleaning} \label{sec:cleaning}
	Focused specifically on the task of point cloud cleaning in heritage scenes

	\subsection{Problem}
	Characterize heritage scenes
	\begin{itemize}
		\item Large scans
		\item many scans
		\item Non uniform density
		\item Large point sets
		\item Hard to distinguish trees from walls
	\end{itemize}

	\subsection{Existing systems}
		\begin{itemize}
			\item Z\&Y
			\item Cyclone
			\item Pointools
			\item Meshlab
			\item VR Mesh Studio
			\item Carlson Pointcloud
			\item 3D Reshaper
			\item Terrascan
		\end{itemize}

	\subsection{Evaluation of existing systems}
	We should look at existing systems in terms of a testing framework
	Evaluate their tools
	Evaluate user interface
	\begin{itemize}
		\item navigation: camera vs object move
		\item tool set (what tools are available)
		\item license
		\item 2d/3d editing
		\item extensibility (why did I not use it)
	\end{itemize}

		

\section{Conclusion}
%What do I conclude from all this and lead into the next chapter.	



\section{Review of literature}

\cite{Spina2010} Cultural heritage segmentation

