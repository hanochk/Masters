\chapter{Background} \label{ch2}

\section{Cultural heritage preservation}
	What is it and why is it important?
	\subsection{Zamani project}
	\subsection{CyArk Project}

\section{Laser scanning}
	\subsection{Processing pipeline}
		\subsubsection{Data aquisition}
			\begin{itemize}
			\item Time of flight scanners
			\item Triangulation scanners
			\item Structure from motion
			\item Scanner resolution increasing over time
			\item Data
			\begin{itemize}
				\item Point coordinates
				\item Non uniform density
				\item Reflectance
				\item RGB
				\item Scanner position
			\end{itemize}
			\end{itemize}


		\subsubsection{Registration}
			\begin{itemize}
			\item 2.5D to full 3D
			\item Iterative closest points
			\end{itemize}
		
		\subsubsection{Cleaning}
			\begin{itemize}
			\item Time consuming
			\item Classification task
			\item Types of noise
				\begin{itemize}
				\item Averaged pixel on edges
				\item Static noise
				\item Moving noise
				\end{itemize}
			\item No points are modified
			\item Focus on interactivity and accuracy
			\item Manual vs automated tools
			\item Tradeoffs between cleaning before and after registration
			\end{itemize}

		\subsubsection{Final steps}
			\begin{itemize}
			\item Surface reconstruction (Mention)
			\item Hole filling (Mention)
			\item Texturing (Mention)
			\end{itemize}


\section{Pointcloud Cleaning}
	\subsection{Existing systems}
		\begin{itemize}
		\item Z&Y
		\item Cyclone
		\item Pointools
		\item Meshlab
		\item VR Mesh Studio
		\item Carlson Pointcloud
		\item 3D Reshaper
		\item Terrascan
		\end{itemize}

	\subsection{Evaluation of existing systems}
		\begin{itemize}
		\item None of the automated methods work well for trees
		\item Proprietary with the exception of Meshlab
		\end{itemize}		

\section{Image segmentation}
	Scans are simply a depth maps so 2D image techniques should be investigated
	\subsection{Image gradients}
	\subsection{Edge detection}
	\subsection{Blob extraction}
	\subsection{Controur extraction}

\section{Laser scan segmentation}
	\subsection{Point features}
		\subsubsection{Normals}
			Basic building block for other features
			\begin{itemize}
			\item Ways of performing normal estimation
			\item Neighbour search
				\begin{itemize}
				\item KD Trees
				\item Scan grid
				\end{itemize}
			\item Cost vs Quality
			\item Dealing with noise
			\item Used with machine learning algorithms
			\end{itemize}
				
		\subsubsection{Fast point feature histograms (FPFH)}
			\begin{itemize}
			\item Good discrimination for primitive geometric objects
			\item Costly to calculate
			\end{itemize}
			
		\subsubsection{Principle components}
			\begin{itemize}					
			\item Good for finding edges and planes
			\item Solves part of the inverse problem
			\end{itemize}
			

	\subsection{Region growing}
		\begin{itemize}
		\item Grow neighbourhood from seedpoint
		\item Add neighbour if it satisfies some similatrity criteria
		\item Point feature can be used to determine similarity
		\end{itemize}

	\subsection{K-means clustering}
		Problems with non uniform denisty

	\subsection{Graph cuts}
		\begin{itemize}
		\item Binary classification
		\item Encode point similatity edges
		\item Shown to work well with arbitrary objects
		\item Results are parameter dependend
		\end{itemize}

	\subsection{Machine learning}
		\begin{itemize}
		\item Used in navigation & aerial scans
		\item Support vector machines
		\item Markov models
		\item Conditional random fields
		\iten Requires training
		\end{itemize}
		
	\subsection{Evaluation of literature}
		\begin{itemize}
		\item FPFH seems to discriminate well for primitive geometry. Might be a useful featature to use in tree classification.
		\item Machine learning has been used to classify vegitation. It however requires training and classified datasets are rare. Possibility to train a classifier on the on scans in the application. Unpredictable accuracy.
		\item Graph cuts have been shown be effective for automated object classification. Augmented with more edge information accuracty may be improved with minimal user interaction.
		\end{itemize}

\section{Evaluation techniques}
\begin{itemize}
	What was used in previous work and would it be suitable in this instance?
\end{itemize}
	
\section{User interaction}
	\begin{itemize}
		\item Level of interactvity expected?
		\item Characteristics of interfaces?
		\item Basic expected functionality?
		\item Suitable evaluation techniques
	\end{itemize}
		

\section{Conclusion?}
What do I conclude from all this and lead into the next chapter.		

